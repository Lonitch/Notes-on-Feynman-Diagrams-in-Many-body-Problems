\section{Bird's-Eye View of Diagram Methods in the
Many-Body Problem}
\begin{table}[H]
        \centering
        
\begin{tabular}{p{0.45\textwidth}p{0.45\textwidth}}
\hline 
 \begin{center}
Field theoretic ingredient
\end{center}
 & \begin{center}
Significance in many-body theory
\end{center}
 \\
\hline 
 \begin{center}
(1)Occupation number notation
\end{center}
 & \begin{center}
Expresses arbitrary state of manybody system
\end{center}
 \\
\begin{center}
(2)Creation and destruction operators
\end{center}
 & \begin{center}
Primitive operators out of which all many-body operators are built
\end{center}
 \\
\begin{center}
(3)Single particle propagator (Green's function)
\end{center}
 & \begin{center}
Yields quasi particle energies, particle momentum distribution, particle density, ground energy
\end{center}
 \\
\begin{center}
(4)Vacuum amplitude
\end{center}
 & \begin{center}
Gives ground state energy
\end{center}
 \\
\begin{center}
(5)Two-particle Green's function propagator 
\end{center}
 & \begin{center}
Yields energies of collective excitations, electrical conductivity, other non-equilibrium properties
\end{center}
 \\
\begin{center}
(6)Finite temperature vacuum amplitude
\end{center}
 & \begin{center}
Gives equilibrium thermodynamic properties of system
\end{center}
 \\
\begin{center}
(7)Finite temperature propagator
\end{center}
 & \begin{center}
Yields temperature dependence of properties in (3)
\end{center}
 \\
\hline
\end{tabular}
        
\end{table}