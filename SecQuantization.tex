\section{Occupation Number Formalism (Second Quantization)}
\subsection{Advantages of occupation number formalism}
Since simple things can sometimes get to look pretty formidable in second quantization it is a good idea to understand why many-body physicists all use it. \bluep{The first reason is that it enables us to deal with systems containing a \textbf{variable number of particles}}.It turns out to give an enormous flexibility in the formalism if \textbf{N is allowed to vary in intermediate stages of a calculation and becomes fixed only at the end.} For example, we can put in and remove test particles at will, as in the case of the propagator. Or we can introduce the particle-hole formalism in which the number of particles and holes is variable.

The second reason for the occupation number formalism has to do with the symmetry properties of Fermi and Boss system. Doing things the old
way, we always have to worry about the complicated business of keeping the wave function properly symmetrized. But it turns out that in second quantization, the \textbf{creation and destruction operators obey certain commutation rules which have built into them all the symmetry properties of the system}. By just using these rules we are automatically free from symmetrization headaches.

\subsection{Many-body wave function in occupation number formalism}
Imagine that we are given a system of N identical fermions, which are in general interacting with each other and with an external potential. We have seen that such a system may be described in terms of a set of basis states, $|n_1,...,n_l,...\rangle$ in which the $n_l$ meant $n_l$ particles in the unperturbed single-particle energy eigenstate, $\phi_l$. Actually, \redp{the single-particle states used can be any orthonormal set}. This means that in general, $|n_1,...,n_l,...\rangle$ are not energy eigenstates of either the interacting or the non-interacting system of particles and their choice is determined by convenience. Foe the moment, we will use the $\phi$'s which satisfy the Schrodinger equation:
\begin{equation}\begin{aligned}
H \phi_{k \sigma}(\mathbf{r}, \gamma) &=\epsilon_{k o} \phi_{k o}(\mathbf{r}, \gamma) \\
H=\frac{p^{2}}{2 m}+U(\mathbf{r}) &=-\frac{1}{2 m} \nabla^{2}+U(\mathbf{r}) \\
(\hbar&=1)
\end{aligned}\end{equation}
and $\gamma, \sigma$ are the spin co-ordinate and quantum number respectively. In the case $U(\mathbf{r})=0,$ this has the solutions:
\begin{equation}\begin{aligned}
\phi_{k \sigma}(\mathbf{r}, \gamma) &=\frac{1}{\sqrt{\Omega}} e^{+i \mathbf{k} \cdot \mathbf{r}} \eta_{\sigma}(\gamma) \\
\epsilon_{k} &=\frac{k^{2}}{2 m}(\hbar=1)
\end{aligned}\end{equation}
where $\eta$ is the spin eigenfunction. In general, $\sigma, \gamma$ will be suppressed for brevity, and $\mathbf{k}$ will be short for $\mathbf{k}, \sigma,$ and $\mathbf{r} \equiv \mathbf{r}, \gamma$. 

If there are now N identical non-interacting fermions, the Hamiltonian and Schrodinger equation become
\begin{equation}\begin{array}{c}
H_{0}=\sum_{l=1}^{N} H_{l}, \quad H_{0} \Phi\left(\mathrm{r}_{1}, \ldots, \mathrm{r}_{N}\right)=E \Phi\left(\mathrm{r}_{1}, \ldots, \mathrm{r}_{N}\right) \\
H_{i}=\frac{p^{2}}{2 m}+U\left(\mathrm{r}_{i}\right) ; \quad H_{i} \phi_{k_{i}}=\epsilon_{k_{i}} \phi_{k_{i}}
\end{array}\end{equation}
since the system consists of identical fermions, the wave function must be antisymmetric, i.e., change sign when any two particle co-ordinates are interchanged. This is accomplished by forming a $\Phi$ given by the Slater determinant.
\begin{equation}\Phi_{k_{1} \ldots \ldots k_{N}}\left(\mathrm{r}_{1}, \ldots, \mathrm{r}_{N}\right)=\frac{1}{(N !)^{\frac{1}{2}}} \sum_{P} \gamma_{P} P\left[\phi_{k_{1}}\left(\mathrm{r}_{1}\right) \phi_{k_{2}}\left(\mathrm{r}_{2}\right) \ldots \phi_{k_{N}}\left(\mathrm{r}_{N}\right)\right]\end{equation}
or
\begin{equation}\Phi_{k_{1}, \ldots, k_{N}}\left(\mathbf{r}_{1}, \ldots, \mathbf{r}_{N}\right)=\frac{1}{(N !)^{\frac{1}{2}}}\left|\begin{array}{c}
\phi_{k_{1}}\left(\mathbf{r}_{1}\right) \phi_{k_{1}}\left(\mathbf{r}_{2}\right) \ldots \phi_{k_{1}}\left(\mathbf{r}_{N}\right) \\
\vdots \\
\phi_{k_{N}}\left(\mathbf{r}_{1}\right) \phi_{k_{N}}\left(\mathbf{r}_{2}\right) \dots \phi_{k_{N}}\left(\mathbf{r}_{N}\right)
\end{array}\right|
\label{slater-determinant-2}
\end{equation}
In the first form, $P$ is the permutation operator which interchanges the $\mathbf{r}_{i}$ 's in all possible ways (starting from some standard order), and $\gamma_{P}=-1$ for an odd number of interchanges, and +1 for an even number. The fact that $\Phi=0$ when any two $k_{i}$ 's are equal means that there can't be more than one particle in any state.

A tricky thing about (\ref{slater-determinant-2}) is its sign. For example, in a two-particle system with one particle in state $\phi_{1}$, and the other in $\phi_3$, the wave function is $\Phi_{k_1=1,k_2=3}\equiv\Phi_{13}$ or $\Phi_{k_1=3,k_2=1}\equiv\Phi_{31}$. Since the particles are identical, these represent the same state, but by (\ref{slater-determinant-2}) they differ by a minus sign. To remove this ambiguity, we always write $\Phi$ with the $k$'s in standard order given by:
\begin{equation}\Phi_{k_{1}<k_{2}<\ldots<k_{N}}\end{equation}
A compact way of writing $\Phi$ is
\begin{equation}\begin{aligned}
\Phi_{k_{1}, k_{2}, \ldots, k_{N}}\left(\mathbf{r}_{1}, \mathbf{r}_{2}, \ldots, \mathbf{r}_{N}\right) &=\Phi_{n_{1}, \ldots, n_{i}, \ldots}\left(\mathbf{r}_{1}, \mathbf{r}_{2}, \ldots, \mathbf{r}_{N}\right) \\
&=\left\langle\mathbf{r}_{1}, \mathbf{r}_{2}, \dots, \mathbf{r}_{N} | n_{1}, \dots, n_{i}, \dots\right\rangle
\end{aligned}\end{equation}
It is important to remember that \redp{the $\left|n_{1}, \ldots, n_{1}, \ldots\right\rangle$ are orthogonal and normal because the $\Phi_{k_{1}, \ldots, k_{N}}$ are}, and we may write this in the various equivalent ways
\begin{equation}\begin{array}{l}
\begin{aligned}
&\left\langle n_{1}^{\prime}, n_{2}^{\prime}, \ldots, n_{1}^{\prime}, \ldots | n_{1}, n_{2}, \ldots, n_{1}, \ldots\right\rangle  \\
& \equiv\left(\Phi_{n_{1}^{\prime}, n_{2}^{\prime}\ldots, n_{l}^{\prime}, \ldots} , \Phi_{n_{1}, n_{2}, \ldots, n_{l}, \ldots}\right) \\
& \equiv \int d^{3} \mathbf{r}_{1} \ldots d^{3} \mathbf{r}_{N} \Phi_{n_{1}^{\prime}, n_{2}}^{\prime}, \ldots .\left(\mathbf{r}_{1}, \ldots, \mathbf{r}_{N}\right) \Phi_{n_{1}, n_{2}, \ldots}\left(\mathbf{r}_{1}, \ldots, \mathbf{r}_{N}\right) \\
&=\delta_{n_{1}^{\prime}, n_{1}} \delta_{n_{2}^{\prime}, n_{2}} \ldots \delta_{n_{l}^{\prime}, n_{l}} \cdots .
\end{aligned}
\end{array}\end{equation}